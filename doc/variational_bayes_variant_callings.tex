\documentclass{article}

\usepackage{fancyhdr} % Required for custom headers
\usepackage{lastpage} % Required to determine the last page for the footer
\usepackage{extramarks} % Required for headers and footers
\usepackage{graphicx} % Required to insert images
\usepackage{lipsum} % Used for inserting dummy 'Lorem ipsum' text into the template
\usepackage{amsmath}
\usepackage{amsfonts}

% Margins
\topmargin=-0.45in
\evensidemargin=0in
\oddsidemargin=0in
\textwidth=6.5in
\textheight=9.0in
\headsep=0.25in 

\linespread{1.1} % Line spacing

\DeclareMathOperator*{\argmax}{arg\,max}

\title{Variational Bayes for variant calling}
\author{Daniel Cooke}
\date{}

\begin{document}

\maketitle

Variational Bayes is a method to approximate the posterior distribution of latent variables in a Bayesian Network. The technique is similar in nature to EM, but whereas EM gives point estimates of the model parameters, Variational Bayes gives a proper distribution over these parameters.  Variational Bayes can be contrasted to stochastic approximation methods; while stochastic approximation methods will always converge to the true distribution given enough time, Variational Bayes can never converge to the true distribution.

Here I derive a Variational Bayes model for approximating the posterior distributions of a genotype model for variant calling. First I describe a generalisation to the standard EM model using my notation.

\section{Definitions \& conditions}

\begin{center}
\begin{tabular}{ll}
Constant & Description \\
\hline
$N$ & the number of samples \\
$M$ & the ploidy of each sample \\
$H$ & the number of haplotypes being considerd \\
$K$ & the number of genotypes \\
\hline
\end{tabular}
\end{center}

Note I make the assumption that all samples have the same ploidy, this assumption could be relaxed but the maths doesn't work out quite as nicely.

\begin{center}
\begin{tabular}{ll}
Variable & Description \\
\hline
$\boldsymbol{x}_n$ & the read data for sample $n$ \\
$\boldsymbol{r}_n$ & the number of reads in the $n^{th}$ sample \\
$\boldsymbol{h}$ & the haplotypes \\
$\boldsymbol{\pi}$ & the probabilties (or 'frequencies') of the haplotypes \\
$\boldsymbol{g}$ & a binary variable using a $1$-of-$K$ coding scheme representing the genotypes \\
\hline
\end{tabular}
\end{center}

In particular $g$ satisfies $g_k \in {0, 1}$ and $\sum_{k = 1}^{K} g_i = 1$.

Finally I define a the function $\mu_i(g)$ which gives the number of occurences of haplotype $i$ in genotype $\boldsymbol{g}$.

I will sometimes abuse notation by using $\boldsymbol{g}^k$ to mean the $\boldsymbol{g}$ with the $k^{th}$ element set to $1$.

Then we have the following conditions:

\begin{equation} 
    K \le \binom{M + H - 1}{H - 1}
\end{equation}

\begin{equation} 
    \sum_{i = 1}^{H} \pi_i = 1
\end{equation}

\begin{equation} 
    \sum_{i = 1}^{H} \mu_i(\boldsymbol{g}) = M
\end{equation}

\begin{equation} 
    \neg\exists i,j, i \ne j\in {1..K} \quad \text{s.t} \quad \mu_k(\boldsymbol{g}^i) = \mu_k(\boldsymbol{g}^j) \quad \forall k \in {1..H}
\end{equation}

\section{EM approach}

Here I review the standard EM model used for genotype modelling. This presentation is actually a slighy generalisation of that usually presented in the literature as it allows for any number of alleles (usually restricted to biallelic).

The marginal distribution for $\boldsymbol{g}$ is given by a multinomial distribution by assuming Hardy-Weinberg equilibrium

\begin{equation} 
    p(g_k = 1) = \binom{M}{\mu_1(\boldsymbol{g}^k),...,\mu_H(\boldsymbol{g}^k)} \prod_{i = 1}^H \pi_i^{\mu_i(\boldsymbol{g}^k)}
\end{equation}

Note due to the $1$-of-$K$ representation for $\boldsymbol{g}$ this can also be written as

\begin{equation} 
    p(\boldsymbol{g}) = \prod_{k = 1}^K \left[\binom{M}{\mu_1(\boldsymbol{g},...,\mu_H(\boldsymbol{g})} \prod_{i = 1}^H \pi_i^{\mu_i(\boldsymbol{g})}\right]^{g_k}
\end{equation}

Using an abuse of notation $h \in \boldsymbol{g}$ to mean the haplotypes in $\boldsymbol{g}$, the data distribution is given by

\begin{equation} 
    p(\boldsymbol{x} | \boldsymbol{g}) = \prod_{i = 1}^{r} \frac{1}{M} \sum_{h \in \boldsymbol{g}} p(r_i | h)
\end{equation}

where $p(r_i | h)$ is determined by an HMM. The marginal data distribution is then

\begin{equation} 
    p(\boldsymbol{x}) = \sum_{\boldsymbol{g}} p(\boldsymbol{g}) p(\boldsymbol{x} | \boldsymbol{g}) = \sum_{k = 1}^{K}  \binom{M}{\mu_1(\boldsymbol{g}^k),...,\mu_H(\boldsymbol{g}^k)} \prod_{i = 1}^H \pi_i^{\mu_i(\boldsymbol{g}^k)} \prod_{i = 1}^{r} \frac{1}{M} \sum_{h \in \boldsymbol{g}} p(r_i | h)
\end{equation}

The posterior probabilities, often called responsabilities, of $\boldsymbol{g}$ given read data $\boldsymbol{x}$ is given by Bayes theorem:

\begin{align} 
    \gamma(g_k) &\equiv p(g_k = 1 | \boldsymbol{x}) \\
    &= \frac{p(g_k = 1) p(\boldsymbol{x} | g_k = 1)}{\sum_{j = 1}^{K} p(g_j = 1) p(\boldsymbol{x} | g_j = 1)}
\end{align}

For the 'E' step of the EM algorithm, we need to evaulate the expectation of the full log-density, to do this we esentially treat each sample as an individual data point. Viewed this way the entire model can roughly be seen to be a clustering model, where samples are assigned to genotype clusters. Let $\boldsymbol{X} = {\boldsymbol{x}_1, ..., \boldsymbol{x}_N}$ and $\boldsymbol{G} = {\boldsymbol{g}_1, ..., \boldsymbol{g}_n}$ (note $\boldsymbol{g}_i$ is different from $g_i$), then the full joint density can be written as:

\begin{equation} 
    p(\boldsymbol{X}, \boldsymbol{G} | \boldsymbol{\pi}) = \prod_{n = 1}^N \prod_{k = 1}^K \left[ \binom{M}{\mu_1(\boldsymbol{g}^k),...,\mu_H(\boldsymbol{g}^k)} \prod_{i = 1}^H \pi_i^{\mu_i(\boldsymbol{g}^k)} \right]^{g_{nk}} \left[ \prod_{i = 1}^{r} \frac{1}{M} \sum_{h \in \boldsymbol{g}} p(r_i | h) \right]^{g_{nk}}
\end{equation}

and so the full log-likelihoood is:

\begin{align} 
    \ln p(\boldsymbol{X}, \boldsymbol{G} | \boldsymbol{\pi}) &= \sum_{n = 1}^N \sum_{k = 1}^K g_{nk} \left\{ \ln \binom{M}{\mu_1(\boldsymbol{g}^k),...,\mu_H(\boldsymbol{g}^k)} \ln \prod_{i = 1}^H \pi_i^{\mu_i(\boldsymbol{g}^k)} \sum{i = 1}^{r} \frac{1}{M} \sum_{h \in \boldsymbol{g}} p(r_i | h) \right\}\\
    &= \sum_{n = 1}^N \sum_{k = 1}^K g_{nk} \left\{ \ln C(\boldsymbol{g}^k) + \sum_{i = 1}^H \mu_i(\boldsymbol{g}^k) \ln \pi_i + \sum_{i = 1}^{r} \ln \sum_{h \in \boldsymbol{g}^k} p(r_i | h) - \ln M \right\}
\end{align}

where $C(\boldsymbol{g}^k)$ is the multinomial coefficient.

As $\boldsymbol{g}$ is a binary variable we have that $\mathbb{E}[\boldsymbol{g}] = \gamma(g_{nk})$ and thus:

\begin{equation} 
    \mathbb{E}_{\boldsymbol{G}} [\ln p(\boldsymbol{X}, \boldsymbol{G} | \boldsymbol{\pi})] = \sum_{n = 1}^N \sum_{k = 1}^K \gamma(g_{nk}) \left\{ \ln C(\boldsymbol{g}^k) + \sum_{i = 1}^H \mu_i(\boldsymbol{g}^k) \ln \pi_i + \sum_{i = 1}^{r} \ln \sum_{h \in \boldsymbol{g}^k} p(r_i | h) - \ln M \right\}
\end{equation}

To maximise w.r.t $\boldsymbol{\pi}$ note we must have $\sum_{i = 1}^H \pi_i = 1$, and so introducing Langrange multipliers we maximise:

\begin{equation} 
    \zeta(\boldsymbol{\pi}) = \mathbb{E}_{\boldsymbol{G}} [\ln p(\boldsymbol{X}, \boldsymbol{G} | \boldsymbol{\pi})] + \lambda \left( \sum_{i = 1}^H \pi_i - 1 \right)
\end{equation}

Differentiating w.r.t $\pi_i$ gives:

\begin{equation} 
   \frac{\partial \zeta}{\partial \pi_i} = \sum_{n = 1}^N \sum_{k = 1}^K \gamma(g_{nk}) \mu_i(\boldsymbol{g}^k) \frac{1}{\pi_i} + \lambda
\end{equation}

equating to $0$ and summing over $i$ gives:

\begin{align} 
    \sum_{i = 1}^H \sum_{n = 1}^N \sum_{k = 1}^K \gamma(g_{nk}) \mu_i(\boldsymbol{g}^k) &= -\lambda \sum_{i = 1}^H \pi_i \\
      \sum_{n = 1}^N \sum_{k = 1}^K \gamma(g_{nk}) \sum_{i = 1}^H  \mu_i(\boldsymbol{g}^k) &= -\lambda \\
      \sum_{n = 1}^N \sum_{k = 1}^K \gamma(g_{nk}) M &= -\lambda \\
      \lambda &= -NM
\end{align}

and so we have:

\begin{equation} 
   \pi_i = \frac{1}{NM} \sum_{n = 1}^N \sum_{k = 1}^K \gamma(g_{nk}) \mu_i(\boldsymbol{g}^k)
\end{equation}

This intuitively makes sense as the expected number of occurences of haplotype $h_i$ under the posterior for $\boldsymbol{G}$.

\section{Variational Bayes}

Using EM we do not get a proper probability distribution over the parameters $\boldsymbol{\pi}$, and thus it is difficult to make proper downstream inferences. The optimal solution would be to use a full Bayesian model by introducing priors over $\boldsymbol{\pi}$ (and as we will see later, possibly also over the hyperparameters of $\boldsymbol{\pi}$).

This leads to problems as even if we choose the conjagte prior for $\boldsymbol{\pi}$, a Dirichlet distribution, the problem is still intractable as we must take the expectation over a sum of Dirichlet-Multinomials. This could be solved using stochastic approximation methods, but these methods are probably not ideal here as they can be slow to converge to the true density, and we also introduce a non-determinsm in the call set that is better avoided.

There are another set of methods for approximating posterior distributions that do not suffer the above problems called Variational Bayesian methods. The idea of these methods is to deterministically approximate the posterior distribution by enforcing a certain factorisation of the posterior distribution that necessarily introduces some independence assumptions. The factor distributions are then chosen to maximise the 'similarity' to the true posterior. While these methods are usually quick to converge, unlike stochastic methods they can never converge to the true distribution.

\subsection{Introduction to variational Bayes}

Given a probability model $p(\boldsymbol{X}, \boldsymbol{Z})$ where $\boldsymbol{X}$ is observed and $\boldsymbol{Z}$ are latent, the true posterior density, $p(\boldsymbol{Z} | \boldsymbol{X})$, can be approximated with another distribution, $q(\boldsymbol{Z})$ subject to some measure of similarity. A natural choice of similairty is the Kullback–Leibler divergence:

\begin{equation}
\label{eq:kl}
   \text{KL} (q\; ||\; p) = -\int q(\boldsymbol{Z}) \ln \frac{p(\boldsymbol{Z} | \boldsymbol{X})}{q(\boldsymbol{Z})} d\boldsymbol{Z}
\end{equation}

This measures the additional amount of information (in nats) required to generate codes from $q$ rather than $p$, it satisfies $\text{KL}(q\; ||\; p) \ge 0$, with equality when $p = q$. So we actually try to minimise this quanitity.

We now partition the latent variables $\boldsymbol{Z}$ into a set of disjoint groups denoted by $\boldsymbol{Z_i}$ where $i = 1, ..., M$ and assume that the $q$ distribution factorizes into a product of these groups, i.e.

\begin{equation}
\label{eq:q}
  q(\boldsymbol{Z}) = \prod_{i = 1}^M q_i(\boldsymbol{Z_i})
\end{equation}

This is the only assumption made, in particular the functional for of each $q_i$ is not contrained.

The idea is then to consider each group in turn, and average over the other groups. Formally this is a problem that can be solved using calculus of variations, but it's easy to see by substituting (\ref{eq:q}) into (\ref{eq:kl}) and seperating one group $\boldsymbol{Z_j}$:

\begin{align}
    \text{KL}(q\; ||\; p) &= -\int \prod_{i = 1}^M q_i \left\{ \ln p(\boldsymbol{Z} | \boldsymbol{X}) - \sum_i \ln q_i) \right\} d \boldsymbol{Z} \\
    &= -\int q_j \left\{ \int \ln p(\boldsymbol{Z} | \boldsymbol{X}) \prod_{i \ne j} d \boldsymbol{Z_i} \right\} d \boldsymbol{Z_j} + \int q_j \ln q_j d \boldsymbol{Z_j} + \text{const}\\
    &= -\int q_j \mathbb{E}_{i \ne j}[\ln p(\boldsymbol{Z} | \boldsymbol{X})] d \boldsymbol{Z_j} + \int q_j \ln q_j d \boldsymbol{Z_j} + \text{const}\\
    &= \text{KL}(q_j\; ||\; \exp (\mathbb{E}_{i \ne j}[\ln p(\boldsymbol{Z} | \boldsymbol{X})])) + const
\end{align}

Clearly the $q_j$ which minimises this quanity is when $q_j = \exp(\mathbb{E}_{i \ne j}[\ln p(\boldsymbol{Z} | \boldsymbol{X})])$ and therefore we find the optimal $q_j$:

\begin{equation}
\label{eq:q_opt}
q^*_j(\boldsymbol{Z_j}) = \mathbb{E}_{i \ne j}[\ln p(\boldsymbol{Z} | \boldsymbol{X})]
\end{equation}

Note these equations do not represent an explicit solution because they are interdependent. The variational Bayes algorithm therefore proceeds similar to EM by cycling through each group, updating $q^*$, and repeating until convergence. It can be shown that $\text{KL}(q\; ||\; p)$ decreases at each step.

\subsection{The variational Bayes genotype model}

As we are approximating a full Bayesian model, we need to specify a prior density on the parameters for the genotype model $\boldsymbol{\pi}$. Although these will now be considered latent variables under the new Bayesian model, they are still dintinct from the other latent variables as they do not grow in number with the number of data points (i.e. samples). I will therefore continue to refer to them as parameters. 

It makes the analysis vastly simpler if we choose a conjugate prior density. We have already noted the marginal genotype density - under Hardy-Weinberg equilibrium - is Multinomial, it therefore makes most sense to choose the prior density $p(\boldsymbol{\pi})$ to be a Dirichlet distribution:

\begin{equation}
p(\boldsymbol{\pi} | \boldsymbol{\alpha}) = \text{Dir}(\boldsymbol{\pi} | \boldsymbol{\alpha}) = \frac{1}{\text{B}(\boldsymbol{\alpha})} \prod_{i = 1}^H \pi_i^{\alpha_i - 1}
\end{equation}

where B is the multinomial Beta function.

The full joint probability distribution is of the form:

\begin{equation}
p(\boldsymbol{X}, \boldsymbol{G}, \boldsymbol{\pi}) = p(\boldsymbol{\pi} | \boldsymbol{\alpha}) p(G | \boldsymbol{\pi}) p(\boldsymbol{X} | \boldsymbol{G})
\end{equation}

Although this defines a full Bayesian model, we could infact do better by introducing another prior over the hyperparameters, and viewings each parameter $\boldsymbol{\pi}$ as a draw from a 'population' density, this would constitute a hierarchical Bayesian model (the haplotype probabilities are automatically updated for each sample). It may be worth investigating whether this is worthwhile. 

There are only two latent variables, so we must have a factorization of the form:

\begin{equation}
q(\boldsymbol{G}, \boldsymbol{\pi}) = q(\boldsymbol{G})q(\boldsymbol{\pi})
\end{equation}

Using (\ref{eq:q_opt}) we find that

\begin{align}
    \ln q^*(G) &= \mathbb{E}_{\boldsymbol{\pi}}[\ln p(\boldsymbol{X}, \boldsymbol{G}, \boldsymbol{\pi})] + const\\
    &= \mathbb{E}_{\boldsymbol{\pi}}[\ln p(\boldsymbol{G} | \boldsymbol{\pi})] + \mathbb{E}_{\boldsymbol{\pi}}[p(\boldsymbol{X} | \boldsymbol{G})] + const\\
    &= \mathbb{E}_{\boldsymbol{\pi}}\left[ \sum_{n = 1}^N \sum_{k = 1}^K g_{nk} \left\{ \ln C(\boldsymbol{g}^k) + \sum_{i = 1}^H \mu_i(\boldsymbol{g}^k) \ln \pi_i \right\} \right] \notag\\ &\phantom{{}=1} + \mathbb{E}_{\boldsymbol{\pi}}\left[ \sum_{n = 1}^N \sum_{k = 1}^K g_{nk} \left\{ \sum_{i = 1}^{r_n} \ln \sum_{h \in \boldsymbol{g}^k} p(r_{ni} | h) - \ln M \right\} \right] + const\\
    &= \sum_{n = 1}^N \sum_{k = 1}^K g_{nk} \left\{ \ln C(\boldsymbol{g}^k) + \sum_{i = 1}^H \mu_i(\boldsymbol{g}^k) \mathbb{E}_{\boldsymbol{\pi}}[\ln \pi_i] + \sum_{i = 1}^{r_n} \ln \sum_{h \in \boldsymbol{g}^k} p(r_{ni} | h) - \ln M \right\} + const\\
    &= \sum_{n = 1}^N \sum_{k = 1}^K g_{nk} \ln \rho_{nk} + const
\end{align}

where we have defined

\begin{align}
    \ln \rho_{nk} &= \ln C(\boldsymbol{g}^k) + \sum_{i = 1}^H \mu_i(\boldsymbol{g}^k) \mathbb{E}_{\boldsymbol{\pi}}[\ln \pi_i] + \sum_{i = 1}^{r_n} \ln \sum_{h \in \boldsymbol{g}^k} p(r_{ni} | h) - \ln M\\
    &= \ln C(\boldsymbol{g}^k) + \sum_{i = 1}^H \mu_i(\boldsymbol{g}^k) (\psi(\alpha)j - \psi(\alpha_0)) + \sum_{i = 1}^{r_n} \ln \sum_{h \in \boldsymbol{g}^k} p(r_{ni} | h) - \ln M\\
    &= \ln C(\boldsymbol{g}^k) + \sum_{i = 1}^H \mu_i(\boldsymbol{g}^k) \mathbb{E}_{\boldsymbol{\pi}}[\ln \pi_i] + \sum_{i = 1}^{r_n} \ln \sum_{h \in \boldsymbol{g}^k} p(r_{ni} | h) - \ln M\\
    &= \ln C(\boldsymbol{g}^k) + \sum_{i = 1}^H \mu_i(\boldsymbol{g}^k) \psi(\alpha_i) - \psi(\alpha_0)\sum_{i = 1}^H \mu_i(\boldsymbol{g}^k) + \sum_{i = 1}^{r_n} \ln \sum_{h \in \boldsymbol{g}^k} p(r_{ni} | h) - \ln M\\
    &= \ln C(\boldsymbol{g}^k) + \sum_{i = 1}^H \mu_i(\boldsymbol{g}^k) \psi(\alpha_i) - M\psi(\alpha_0) + \sum_{i = 1}^{r_n} \ln \sum_{h \in \boldsymbol{g}^k} p(r_{ni} | h) - \ln M
\end{align}

where $\phi$ is the digamma function, $\alpha_0 = \sum_{i = 1}^H \alpha_i$, and $C(\boldsymbol{g}^k)$ is the multinomial coefficient.

Hence we find the optimal density for $\boldsymbol{G}$ is given by

\begin{equation}
q^*(\boldsymbol{G}) = \prod_{n = 1}^N \prod_{k = 1}^K \tau_{nk}^{g_{nk}}
\end{equation}

where we have defined

\begin{equation}
\tau_{nk} = \frac{\rho_{nk}}{\sum_{j = 1}^K \rho_{nj}}
\end{equation}

The sum in the denominator of $\tau$ is a normalisation constant. Next we need to find the optimal distribution for $\boldsymbol{\pi}$. Noting that as $\boldsymbol{g}$ is binary so we have $ \mathbb{E}[g_{nk}] = \tau_{nk}$, we again use (\ref{eq:q_opt}) to find

\begin{align}
    \ln q^*(\boldsymbol{\pi}) &= \ln p(\boldsymbol{\pi}) + \mathbb{E}[\ln p(\boldsymbol{G} | \boldsymbol{\pi})] + const\\
    &= \sum_{i = 1}^H (\alpha_i - 1) \ln \pi_i + \sum_{n = 1}^N \sum_{k = 1}^K \mathbb{E}[g_{nk}]\left\{ \ln C(\boldsymbol{g}^k) + \sum_{i = 1}^H \mu_i(\boldsymbol{g}^k) \ln \pi_i \right\} + const\\
    &= \sum_{i = 1}^H (\alpha_i - 1) \ln \pi_i + \sum_{n = 1}^N \sum_{k = 1}^K \tau_{nk} \left\{ \sum_{i = 1}^H \mu_i(\boldsymbol{g}^k) \ln \pi_i \right\} + const
\end{align}

and therefore

\begin{align}
    q^*(\boldsymbol{\pi}) &= \prod_{i = 1}^H \pi_i^{\alpha_i - 1} \prod_{i = 1}^H \pi_i^{\sum_{n = 1}^N \sum_{k = 1}^K \tau_{nk} \mu_i(\boldsymbol{g}^k)} + const\\
    &= \prod_{i = 1}^H \pi_i^{\alpha_i + \sum_{n = 1}^N \sum_{k = 1}^K \tau_{nk} \mu_i(\boldsymbol{g}^k) - 1} + const
\end{align}

Which we can recognise as a Dirichlet distribution

\begin{equation}
q^*(\boldsymbol{\pi}) = \text{Dir}(\boldsymbol{\pi} | \boldsymbol{\alpha}')
\end{equation}

where the components of $\boldsymbol{\alpha}'$ are given by

\begin{equation}
\alpha'_i = \alpha_i + \sum_{n = 1}^N \sum_{k = 1}^K \tau_{nk} \mu_i(\boldsymbol{g}^k)
\end{equation}

It is worthwhile noting that this variational Bayes model has the same runtime complexity as the EM algorithm.

\subsubsection{Approximate posterior predictive destribution}

We can now use the approximate densities to approximate posterior predictive destributions, which we can use to make inferences about the number of haplotypes in the population. For a vector of haplotype counts $\boldsymbol{z}$, the approximate posterior predictive distribution is

\begin{align}
    p(\boldsymbol{z} | \boldsymbol{X}, \boldsymbol{\alpha}) &= \int p(\boldsymbol{\pi} | \boldsymbol{X}, \boldsymbol{\alpha}) p(\boldsymbol{z} | \boldsymbol{\pi}) d\boldsymbol{\pi}\\
    &\approx \int q^*(\boldsymbol{\pi}) p(\boldsymbol{z} | \boldsymbol{\pi}) d\boldsymbol{\pi}\\
    &= \int \text{Dir}(\boldsymbol{\pi} | \boldsymbol{\alpha}) \text{Mul}(\boldsymbol{z} | \boldsymbol{\pi}) d\boldsymbol{\pi}\\
    &= \frac{z_0!}{\prod_{i = 1}^{H} z_i} \frac{\Gamma(\alpha_0)}{\Gamma(z_0 + \alpha_0)} \prod_{i = 1}^H \frac{\Gamma(z_i + \alpha_i)}{\Gamma(\alpha_i)}
\end{align}

where $z_0 = \sum_{i = 1}^H z_i$ and $\alpha_0 = \sum_{i = 1}^H \alpha_i$.

\subsubsection{Finding the most probable allele counts}

We could then find the most probable essemble of haplotypes in the samples by find a MAP estimate for $\boldsymbol{z}$, given the total number of haplotypes is $S = NM$.

\begin{align}
    \hat{\boldsymbol{z}}_{MAP} &= \argmax_{\boldsymbol{z}} \check{p}(\boldsymbol{z} | \boldsymbol{X}, \boldsymbol{\alpha}, S = NM)\\
    &= \argmax_{\boldsymbol{z}} \frac{\check{p}(\boldsymbol{z} | \boldsymbol{X}, \boldsymbol{\alpha})p(S = NM | \boldsymbol{z})}{p(S = NM)}\\
    &= \argmax_{\boldsymbol{z}} \check{p}(\boldsymbol{z} | \boldsymbol{X}, \boldsymbol{\alpha})p(S = NM | \boldsymbol{z})\\
    &= \argmax_{\boldsymbol{z} \; \text{s.t} \; \boldsymbol{z} \in \mathbb{N},\; \sum_k z_k = NM} \frac{NM!}{\prod_{i = 1}^{H} z_i} \frac{\Gamma(\alpha_0)}{\Gamma(NM + \alpha_0)} \prod_{i = 1}^H \frac{\Gamma(z_i + \alpha_i)}{\Gamma(\alpha_i)}
\end{align}

In theory this is an integer programming problem which is NP-hard, but if $S$ is small then it is not too much work to find the optimal value.

\subsubsection{Genotype posteriors}

The posterior genotype density for each sample is found using empirical Bayes, i.e. by taking the posterior haplotype densities as priors. A better approach would be to design a full hierarchical model.

\begin{equation}
p(g_{nk} = 1 | \boldsymbol{X}, \boldsymbol{\alpha}) = \frac{p(g_k = 1 | \boldsymbol{\pi}) p(\boldsymbol{x}_n | g_k = 1)}{\sum_{j = 1}^K p(g_j = 1 | \boldsymbol{\pi}) p(\boldsymbol{x}_n | g_j = 1)}
\end{equation}

\end{document}
